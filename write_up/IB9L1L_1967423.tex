% !TEX TS-program = pdflatexmk
% Dissertation write up - 1967423

\documentclass[11pt]{article}

% Packages
\usepackage[margin = 1in, includefoot]{geometry}
\usepackage{setspace}
\usepackage{helvet}
\usepackage{natbib}

% Bib style
\bibliographystyle{agsm}

% Global commands
\renewcommand{\familydefault}{\sfdefault} % Set font as Helvetica (close enough to Arial)
\onehalfspacing % 1.5 line spacing

% Cover page details
\title{Trends in the pervasiveness of Central Bank conspiracy theories and its consequences for monetary policy: Mining ``Bank of England'' Twitter}
\author{Course code: IB9L1L\\
	    Candidate: 1967423\\
	    Wordcount: xxxx\footnote{According to TexCount Version 3.1, available at: \texttt{http://app.uio.no/ifi/texcount/}}}
\date{\today}

\begin{document}

\begin{titlepage}
	\centering
	\maketitle
\abstract{blah blah blah blah blah blah blah blah blah blah blah blah blah blah blah blah blah blah blah blah blah blah blah blah blah blah blah blah blah blah blah blah blah blah blah blah blah blah blah blah blah blah blah blah blah blah blah blah blah blah blah blah blah blah blah blah blah blah blah blah blah blah blah blah blah blah blah blah blah blah blah blah blah blah blah blah blah blah blah blah blah blah blah blah blah blah blah blah blah blah blah blah blah blah blah blah blah blah blah} 

\vspace{1in}

This is to certify that the work I am submitting is my own. All external references and sources are clearly acknowledged and identified within the contents. I am aware of the University of Warwick regulation concerning plagiarism and collusion.
No substantial part(s) of the work submitted here has also been submitted by me in other assessments for accredited courses of study, and I acknowledge that if this has been done an appropriate reduction in the mark I might otherwise have received will be made.
	
	\end{titlepage}
	
\section{Introduction} \label{Section: Introduction}

\section{Central Banks, Populism, and Conspiracy Theories} \label{Section: Central Banks, Populism, and Conspiracy Theories}

\subsection{Populism and Central Banks}

The recent proliferation of populism has caused the concept to be among the hottest, and yet most controversial, topics in political science \citep{brubaker2020populism}. There is a lack of consensus on its definition, key characteristics, empirical extensions and even whether it represents a positive, negative or neutral political force \citep{bergmann2020populism, brubaker2020populism, caiani2019understanding}. The importance of populism is emphasised by authors suggesting it constitutes the second dimension to the left-right political spectrum \citep{koch2021varieties}. Some go so far as to say it has replaced the historic left-right cleavage \citep{de2018populism}. With reference to central banks, it is clear that populism transcends the left-right divide and that populist-style rhetoric can be found on both sides of the political spectrum. For example, the right-wing nationalist politician Donald Trump has referred to central bankers at the Federal Reserve as ``boneheads'' and urged them to keep interest rates low. His rhetoric is clearly nationalist and anti-elite in nature, stating on Twitter that ``it is incredible that with a very strong dollar and virtually no inflation, the outside world blowing up around us, Paris is burning and China way down, the Fed is even considering yet another interest rate hike. Take the Victory!'' \cite[pg.~9]{binder2021technopopulism}. On the other hand, left-wing presidential candidate Bernie Sanders has also engaged in populist discourse towards the Federal Reserve. In an article from 2015, he suggested members of the Federal Reserve's boards should be reorganised to include ``representatives from all walks of life — including labor, consumers, homeowners, urban residents, farmers and small businesses'' \citep{sanders2015bernie}. What these two examples have in common is a general distrust for the elites and a belief that the economic incumbents do not represent the best of interest of a homogenous `people'.

Why are central banks a target for populist rhetoric? As highlighted by the ex-Governor of the Reserve Bank of India, Raghuram Rajan, ``with their PhDs, exclusive jargon, and secretive meetings in far-flung places like Basel and Jackson Hole, central bankers are the quintessential rootless global elite that populist nationalists love to hate'' \citep{rajan2017central}. That is, central bankers represent the global elite that populists believe are disconnected with the needs of the people. Furthermore, populism is often interpreted as a direct response to the crisis of representation found in liberal representative democracies \citep{de2018populism}. There has been a gradual collapse in the legitimacy of mass parties' claims to broaden and represent their electoral base.  As \cite{de2018populism} explain, the cleavage between the people and their representatives has widened and the void has been filled with a new approach to representation: bottom-up participation and direct democracy; the legitimation of authoritarian leaders who are the supreme representative for the people's interests; and the delineation between the people and the ``non-people'' composed of the elites. It is no surprise then, that central banks, who were gifted independence from direct political interference by the very parties who are now experiencing a crisis of legitimacy, are under fire from populists who are proponents of direct democracy and the will of the people. 

Furthermore, it is argued that the recent growth of populism has in part been cause by structural changes in the real economy and emphasised by the global financial crisis. \citep{gnan2020populism} highlight several economic factors including: (a) technological change leaving less educated, older and rural groups unable to adjust; (b) economic liberalisation, migration and labour market reforms attacking incumbents’ rents and wages; (c) globalisation raising fears of competition from low-cost countries; and (d) the rise of income and wealth inequality. Regulatory failings leading up to the financial crisis \citep{turner12017did} undermined the legitimacy of policy makers, regulators and central banks alike. Central banks are a symbol of the economic dissatisfaction that is fuelling populist movements across the globe.

In summary, central banks are (a) globalist institutions promoting technical expertise; (b) making decisions with limited direct democratic accountability and whose accountability is to politicians that have lost legitimacy themselves; and (c) symbols of structural economic changes and the global financial crisis. These three factors mean it is no supriose that independent central banks find themselves in the firing line from populist leaders worldwide.

\section{Studying ``Bank of England'' Twitter} \label{Section: Studying ``Bank of England'' Twitter}

\subsection{Methodological Approach} \label{Subsection: Methodological Approach}
\subsection{Data collection and summary} \label{Subsection: Data collection and summary}
\subsection{Topic Modelling} \label{Subsection: Topic Modelling}
\subsection{Dictionary-based classification} \label{Subsection: Dictionary-based classification}
\subsection{Similarity index} \label{Subsection: Similarity index}

\section{Discussion} \label{Section: Discussion}

\clearpage
\bibliography{IB9L1L_bib.bib}

\end{document}